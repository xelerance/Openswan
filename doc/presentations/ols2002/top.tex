%\documentclass[twocolumn]{article}
\documentclass{article}
\usepackage[dvips]{graphicx}
%\usepackage{ols

% You have to do this to suppress page numbers.  Don't ask.
\thispagestyle{empty}

\usepackage{html} 

\author{Michael C. Richardson\\
	Sandelman Software Works Inc.\\
{\normalsize {\tt mcr@sandelman.ottawa.on.ca} http://www.sandelman.ca/mcr} \\}

%don't want date printed
\date{}

%\date{April 20, 2002}

\title{\Large \bf Automatic Regression testing of network code: User-Mode Linux and FreeSWAN.}

\begin{document}
\maketitle\newpage

\begin{abstract}

The Linux FreeSWAN project (IPsec for Linux) produces rather complicated
networking code. The successful application of the protocol results in all
network data being encrypted. The use of dynamic keying means that it nearly
impossible for an observer (even a trusted one trying to test) to know
what is going on. The need for multiple systems (often as many as 6) to be
properly configured creates an environment nearly impossible to test
regularily.

The emergence of virtual machine technology, particularly, User Mode Linux,
has provided a solution to the testing problem: create as many virtual
machines as needed and control them using standard testing scaffolding
technology: expect(1). This paper describes the scaffolding and the resulting
testing regime which is used. 

The focus is around a modified network switch emulator, "uml\_netjig" which
provides the ability to play and capture network packets through a single
User-Mode Linux virtual machine. 

A second iteration of this tool is also described, combining more complicated 
expect scripts, and a command mode for uml\_netjig, permitting coordination of 
the multiple virtual machines that are needed when doing fully negotiated
IPsec sessions.

\end{abstract}

\section{Background: What is this about}

The Linux FreeS/WAN project is a funded project. It has the mandate to
produce an IPsec implementation for Linux. The ultimate goal of this effort
is to provide systems and software to permit citizens for the world to keep
all of their Internet traffic private.
\footnote{See \htmladdnormallink{http://www.freeswan.org/}{http://www.freeswan.org/}}.

Testing networking protocols is often difficult. By definition there is at
least one network involved and often several independant systems attached to
the network. 

With many application layer network protocols (e.g. http) one can cheat - the 
network is the virtual ``loopback'' device, and multitasking permits both
ends of a protocol to run on the same host. It is therefore common to see
people doing all sorts of network development using a garden variety notebook.

The situation is not the same for transport and network layer protocols such
as IP, TCP and IPsec. These layers of the protocol are more fundamental. They 
are typically implemented inside a system kernel. This makes development work
as difficult as generic kernel work.

If one is to test them on one's notebook or desktop, one risks putting ones
own development environment at risk. It is common experience that doing
kernel development is much easier with at least two machines - one machine is 
crashed every ten minutes and the other machine is used as the development
host. The split between development and testing is much better understood in
embedded system work - the machine under test is often of a totally different 
type than the development machine. Historically, the machine under test (a
VCR or a modem) is incapable of even running a development environment. 

Network protocol development work is further complicated by the need to have
more than one machine involved. 

\subsection{eXtreme Programming}

The growing discipline of eXtreme Programming\cite{XP}
has a number of fundamental principles
\begin{itemize}
\item rapid feedback
\item assume simplicitiy
\item incremental change
\item embracing change
\item quality work
\end{itemize}

\cite{XP} goes on to explain that the fundamental activities are coding,
testing, listening and designing. XP tries to reach the point where one
writes the test cases before the code. To do this, the cost (in effort and time) of
testing must be reduced such that all tests can be run frequently - several
times a day if possible. This very rapid feedback reduces the risk of
introducing problems - permitting developers to program more efficiently and
with more confidence 

This paper describes the typical requirements for doing network testing. The
reasons why it is expensive and why it is difficult to automate are
explained. 
Our solution, uses User-Mode-Linux to turn machines into processes. The
scaffolding is then used to control these processes, to put them through
their paces on a regular basis.

No solution is perfect on the first pass - XP actually encourages partial
solutions to be implemented and feedback to be received - so we describe our
second pass, which at the time of writing, is still in the design phase.






\section{How to test with physical hardware}

\begin{figure}[ht]
\includegraphics[height=4in,width=5in]{testsetup1.eps} 
\caption{Basic Physical Network configuration}
\label{basicnet}
\end{figure}

The basic network is shown in Figure \ref{basicnet}. The taxonomy for our test setup
is that the Sun rises in the east and the sun sets in the west. Thus one can
easily remember where each host is. 

{\scshape east} and {\scshape west}, shown with firewall icons, are FreeS/WAN IPsec gateway boxes.

{\scshape sunrise} and {\scshape sunset} are just ordinary hosts whose traffic
will be protected by their respective gateways.

The machine {\scshape sky} is used to do network analysis (``sniff''). 
There are frequently problems that occur when trying examine the traffic
produced by a machine itself, so a seperate machine to make unbiased
observations if necessary.\footnote{In particular, on Linux 2.2 or lower,
turning on the packet capture mechanism changes the control structures
attached to the traffic and causes faults relating to policy for the keying
channels' control packets. PR\#48 at 
\htmladdnormallink{http://bugs.freeswan.org:81/bugs/gnatsweb.pl}
{http://bugs.freeswan.org:81/bugs/gnatsweb.pl?&database=freeswan&cmd=view&pr=48}
2.4 has solved this problem}

The two gateway boxes are not directly attached, but rather are connected via 
a router. There are two reasons for this:
\begin{itemize}
\item the current implementation of FreeS/WAN requires a default route to
	operate correctly.
\item a common operational issue is with links where the Maximum Transmission 
	Unit (MTU) is restricted, and this router provides a place to cause
	such an impairment\footnote{FreeS/WAN has adopted the term
	``impairment'' to denote any challenges which are introduced to a
	system or network to permit another part of the system to be tested}
\end{itemize}

This setup is very representative of the typically deployed scenario for
FreeS/WAN systems in a VPN. It does not cover every single situation - most
of the most difficult to reproduce bugs have occured in other setups. More
machines are needed to create such setups.

Aside from the space and cost involved in providing each developer with six
machines (it is often the case that {\tt sky} is the developers desktop),
there are a number of other factors that make this difficult.  

The major problem is maintaining this setup. There are many machines with
many files that must be maintained. The systems must be kept up-to-date so
that the latest kernels can be tested, yet at the same time, testing against
older kernels is necessary. Different distributions need to be tested. The
combinatorics are quite high.

The other major problem is work environment. Sitting in a room with six
computers is a lot of noise. Getting access to each systems' console is
difficult (one can not rely upon network logins!). If a monitor is attached
to each system (vs a monitor switch), then the developer probably gets too
much exersize.

One answer to this is serial consoles. See figure
\ref{basicconsole}. Terminals attached to serial ports was the primary 
way that people used Unix until the advent of the X-terminal, and Linux 
continues this grand tradition.

\begin{figure}[hf]
\includegraphics[height=4in,width=5in]{testsetup2.eps} 
\caption{Basic Network with console access}
\label{basicconsole}
\end{figure}

One simply puts the following in {\tt /etc/lilo.conf}:
\begin{verbatim}
serial=0,38400n

...
image=/boot/vmlinuz-2.4.18-6mdk
        label=linux2418
        root=/dev/hda1
        append=" devfs=mount console=ttyS0,38400 console=tty0"
        read-only

\end{verbatim}

The console then appears on both ``COM1'' and on the VGA screen. In this
situation, the machines may be located in another room, connected to a
console server. One logins from one's (quiet) desktop to the console server,
accessing each machine via a serial port. Serial interfacesd are readily
available with either PCI or USB interfaces. This makes building a 6 port
console server rather easy.

The developer now has ready access to each machine, can reboot each machine,
select different kernels and can configure it without even having networking on.
In addition, kernel panics (``kernel oops'') or other strange output on the
console can be cut and pasted into emails, etc..

\subsection{Still challenging to test}

The serial consoles do not solve the other problems - managing the very many 
different configurations, or coordinating the systems to perform a test case.

The author has used such a setup for many years with many Unix operating
systems. Using the ``expect'' program and the serial consoles one can
automate some of the tests. Some of tests are harder to deal ones - ones that 
fail can cause the system to hang - this will require operator
intervention. Further use of more hardware can solve this problem as well -
relays can toggle reset switches or even power cycles. 

The result, however, is a very complicated testing environment - it can take
weeks to configure it, and mere hours to break. There is far too much
specialized hardware involved, not to mention the software.

There is a better way which will be described, but first, the requirements
for the testing environemnt will be examined in a bit more detail.






	

\section{What do we really need}

\subsection{A brief primer on IPsec}

IPsec\cite{RFC2411},\cite{RFC2401} consists of three transport layer protocols:
AH\cite{RFC2402}, ESP\cite{RFC2406} and IPcomp\cite{RFC2507}. There is one
management protocol in existence at this time,
ISAKMP\cite{RFC2408}/IKE\cite{RFC2407},\cite{RFC2409}. 

These transport protocols can be applied to upper layers of TCP, UDP, or any
other transport protocol. When the upper layer is the ``IPIP''\cite{RFC2003},
then the protocol is said to be in ``tunnel'' mode. For most Virtual Private
Network (VPN) usages, tunnel mode is the preferred method since it hides the
origina source/destination address. VPNs are often treated as being virtual
leased lines.

Each of the transport protocols provide session layer encryption. They are
referred to as ``security associations''. These are unidirectional concepts - 
a pair is usually needed for bidirectional communications.

\subsubsection{Authentication Header (AH)}

The Authentication Header provides origin authentication and integrity of the 
headers and of the data portion. No privacy is provided.

\subsubsection{Encapsulating Security Payload (ESP)}

The ESP header provides origin authentication, integrity and optional privacy 
of the data portion only. Normally, this privacy option is provided by
encryption, but the specification permits a ``null'' encryption to be used in 
some circumstances.

\subsubsection{IP compression header (IPcomp)}

A good encryption algorithm produces cyphertext that is evenly
distributed. This makes it difficult to compress. If one wishes to compress
the data it must be done prior to encrypting. The IPcomp header provides for this.

One of the problems of tunnel mode is that it adds 20 bytes of IP header,
plus 28 bytes of ESP overhead to each packet. This can cause large packets to
be fragmented. Compressing the packet first may make it small enough to avoid 
this fragmentation.

\subsubsection{Internet Security Association Key Management Protocol (ISAKMP)}

ISAKMP is a framework for doing Security Association Key Management. It can,
in theory, be used to produce session keys for many different systems, not
just IPsec.

\subsubsection{Internet Key Daemon (IKE)}

IKE is a profile of ISAKMP that is for use by IPsec. It is often called simply
``IKE''. IKE creates a private, authenticated key management channel. Using
that channel, two peers can communicate, arranging for sessions keys to be
generated for AH, ESP or IPcomp. The channel is used for the peers to agree
on the encryption, authentication and compression algorithms that will be
used. The traffic to which the policies will applied is also agreed upon.

\subsection{Testing KLIPS}

In FreeSWAN, the session layer encryption, security association management
and traffic selection is done by a kernel component called KLIPS (Kernel Level
IP Security). This component can be built as a loadable kernel module or
statically built in. 

As the security associations are unidirectional one can effectively seperate
the encrypt/encapsulate and decrypt/decapsulate operations for testing
purposes.

For ease of thinking, the encryption operations are always done on {\scshape
EAST} and the decryption operations are always done on {\scshape WEST}.

\begin{figure}
\includegraphics[height=3in,width=5in]{klipstest.eps} 
\caption{How to test KLIPS}
\label{klipstest}
\end{figure}

As indicated in figure \ref{klipstest}, a source of plaintext packets is
needed, a way to examine the ciphertext packets is needed, and a way to
configure the system is needed. In the physical setup of the previous
section, the source of plaintext packets is provided by the machine {\scshape
SUNRISE}, and the examination of the packets is provided by {\scshape SKY}.  

A typical initialization script for KLIPS is shown below:

\begin{verbatim}
#!/bin/sh
TZ=GMT export TZ

ipsec spi --clear
ipsec eroute --clear

enckey=0x4043434545464649494a4a4c4c4f4f515152525454575758
authkey=0x87658765876587658765876587658765

ipsec klipsdebug --set pfkey
ipsec klipsdebug --set verbose

ipsec spi --af inet --edst 192.1.2.45 --spi 0x12345678 --proto esp --src 192.1.2.23 --esp 3des-md5-96 --enckey $enckey --authkey $authkey

ipsec spi --af inet --edst 192.1.2.45 --spi 0x12345678 --proto tun --src 192.1.2.23 --dst 192.1.2.45 --ip4

ipsec spigrp inet 192.1.2.45 0x12345678 tun inet 192.1.2.45 0x12345678 esp 

ipsec eroute --add --eraf inet --src 192.0.2.0/24 --dst 192.0.1.0/24 --said tun0x12345678@192.1.2.45

ipsec tncfg --attach --virtual ipsec0 --physical eth1
ifconfig ipsec0 inet 192.1.2.23 netmask 0xffffff00 broadcast 192.1.2.255 up

# magic route command
route add -host 192.0.1.1 gw 192.1.2.45 dev ipsec0

ipsec look
\end{verbatim}

The term SPI means ``Security Parameters Index''. Each security association
is indexed by a SPI. Note that a seperate SPI is setup for the ESP operation
and for the tunnel operation. The two are then grouped together. 

The {\tt eroute} (Extended Route) command then selects traffic by source and
destination address for processing by the afore mentioned
group. \cite{RFC2401} defines other selectors, including TCP and UDP port
numbers, but those selectors are not implemented in KLIPS at this time.

The {\tt tncfg} command attaches the IPsec pseudo to a physical device. This
is necessary in 2.0 and prior kernels to provide a path for the resulting
packets to actually leave the system. Otherwise, the {\tt route} command at
the end can cause packets to loop internally. Eliminating this problem -- 
we refer to it as ``stoopid routing tricks''<tm> -- is the major goal of
revisions to KLIPS.

The {\tt ipsec klipsdebug} commands turn on various debugging output. This
debugging output is important for diagnosing what has really happened when
the system fails.

Finally, the {\tt ipsec look} command produces a short summary of resulting
system setup. The output of this looks like:

\begin{verbatim}
east Tue Apr  2 04:32:28 GMT 2002
192.0.2.0/24       -> 192.0.1.0/24       => tun0x12345678@192.1.2.45 esp0x12345678@192.1.2.45  (0)
ipsec0->eth1 mtu=16260(1500)->1500
esp0x12345678@192.1.2.45 ESP_3DES_HMAC_MD5: dir=out src=192.1.2.23 iv_bits=64bits iv=0x24a4a14e81ee960e alen=128 aklen=128 eklen=192 life(c,s,h)=addtime(9,0,0)
tun0x12345678@192.1.2.45 IPIP: dir=out src=192.1.2.23 life(c,s,h)=addtime(9,0,0)
Kernel IP routing table
Destination     Gateway         Genmask         Flags   MSS Window  irtt Iface
192.0.1.1       192.1.2.45      255.255.255.255 UGH      40 0          0 ipsec0
192.1.2.0       0.0.0.0         255.255.255.0   U        40 0          0 eth1
192.1.2.0       0.0.0.0         255.255.255.0   U        40 0          0 ipsec0
192.0.1.0       192.1.2.45      255.255.255.0   UG       40 0          0 eth1
192.0.2.0       0.0.0.0         255.255.255.0   U        40 0          0 eth0
0.0.0.0         192.1.2.254     0.0.0.0         UG       40 0          0 eth1
\end{verbatim} 

At this point, the system is ready to have packets sent through it. If the
packets match the criteria for the SA, then they will be encrypted with the
provided key.

\subsubsection{KLIPS hassles}

The observant will notice a number of numbers in the above output which were
not in the script: the IV field ({\tt 0x24a4a14e81ee960e}), the lifetime
values (it has been 9 seconds between the SA was created and the look
command occured), and the date.

These variances cause two problems: the console output is not consistent on
every run, and the resulting encrypted packets will have different ciphertext 
on each run. 

%The console output problem can be handled by removing pieces which are not
%relevant in a post processing way. 
% (Note the forcing of the time zone at the beginning of the
% script. This was done so that all dates would have the string ``GMT'' in them)
% The difference in the ciphertext can be
%handled by not comparing the ciphertext directly - rather an independant
%program can be used to decre

\subsection{Testing Pluto}

Pluto requires two machines to do any testing. Pluto has to talk to
a peer that understand the ISAKMP/IKE protocol. 

Where IPsec can be configured easily to be in a static, or at least,
predictable state, ISAKMP/IKE can not. The protocol has extensive facilities
to prevent replay attacks, does all of its operations in private and
generates random keys for use by the KLIPS system.

\begin{figure}
\includegraphics[height=2in,width=3in]{plutotest.eps} 
\caption{How to test Pluto}
\label{plutotest}
\end{figure}

Most actions that Pluto performs are done in response to
network traffic. There are three kinds of traffic that causes pluto to
change states:
\begin{itemize}
\item IP traffic caught by KLIPS, that causes PF\_KEY traffic to pluto.
\item another host that wants to initiate keying with this host.
\item replies to DNS queries that pluto sends.
\end{itemize}

In addition, timeouts are important. To test pluto you either have to
configure rather small timeouts - which can cause race conditions, or
one has to be very patient. A way to artificially advance the clock
would help.



\section{The first virtual attempt}

\begin{figure}[ht]
\includegraphics[height=4in,width=5in]{single_netjig.eps} 
\caption{NetJig interface diagram}
\label{netjig}
\end{figure}

User Mode Linux was used to help automate the tests. A user mode linux
is built which contains the required code. A number of
root file systems are prepared by ``make uml''. Each contain the
configuration for a specific node in the network: east, west, sunrise,
sunset, nic and japan.

The apparatus shown arranges to run one of these ``machines'' with
inputs and outputs connected to sources of replayed data.

The network interfaces are managed by ``uml\_netjig'' - a program that is
a variation of User-Mode-Linux's uml\_switch (formerly called uml\_router).
uml\_netjig opens a pair of network sockets for each network interface,
and then starts the expect script.

The TCL expect script shown, host-test.tcl, starts the User-Mode-Linux
machine with options to have it boot into single user mode. This is done using
the standard expect command ``spawn'' - the script can then interact
with the sub-processes standard in and out. As User-Mode-Linux will
connect stdin/stdout as the console of the machine, the script
can control the virtual machine. 

The first thing that the script does it configure the machine. The
enclosing uml\_netjig script is waiting for this script to
exit. host-test.tcl sets up all of the required parameters, and
then it calls fork(2).

The parent exits, while the child continues to manage the console. If the
child were to exit, the virtual machine would have no console and it would
be impossible to capture further output.

When the parent exits, the uml\_netjig can then continue. It opens 
any files that it was been told to read, and plays the packets it finds
back on the public or private interfaces. If told to, the uml\_netjig
process will answer any ARP requests that it sees.

In addition, it records any packets that are emitted by the process
under test and stores them in a file.

When the last packet has been played, then uml\_netjig assumes that
the test is over, and exits. The enclosing script then arranges to shutdown
the user-mode-linux process using uml\_mconsole.

The enclosing script then converts the captured network files (written
in libpcap format) to text using tcpdump, and compares that to a file
of expected values.

The captured console output is also compared. 

Both the libpcap and the console output files are first put through some
scripts to canoncalize them. In the case of console output, this means
removing things like date stamps, kernel versions, ... from the kernel
boot messages. There are also a number of KLIPS things that get in the
way. Where possible, we have either removed them, or regularized them to
make the scripts simpler.

\subsection{Review of first attempt}

The worst problem with this system is that it only manages a single
virtual machine. This means that the first attempt does not test
any automatic keying.

The second worst problem is that it can not perform tests that do not
involve replaying packets. For instance, one can not run a invoke
a process on the console that sends packets and capture them. That part
will work, but since there are no packets to replay, there is no way to
know when the test has completed.

The third problem is that once the host-test.tcl program has detached
itself into the background, no further console control is possible.
This means, in particular, there is no way to gather statistics at the
end of the test (such as number of packets that have gone through, or
current state of the eroute table). There is in fact, no way to even
determine when the test is over.




\section{The second virtual attempt}

To fix the various problems noted before, a rewrite of the uml\_netjig was
done. The rewrite includes a command mode - the result is that the switch
controller is now just another I/O from the expect script.

\begin{figure}[ht]
\includegraphics[height=2in,width=3in]{multi_netjig.eps} 
\caption{NetJig for multiple machines}
\label{netjig2}
\end{figure}

The major thing to note in contrasting this diagram with the previous
diagram is that the expect script is now in control. The expect
script starts up a copy of the switch emulator (uml\_netjig), asks it to
create three virtual switches: eastnet, westnet and publicnet. 

These are connected to the User-Mode-Linux systems in turn. 

Initialization scripts are then run on the two user-mode-linux systems.

At the completion of the initialization, the expect script can then
command the uml\_netjig to play back captured packets on appropriate
interfaces, while recording packets from other interfaces.

In the simplest test case, {\tt basic-pluto-01} packets are played 
on the {\tt eastnet}, addressed from {\tt sunrise} (a node on eastnet),
to {\tt sunset} (a node on westnet). The two machines have negotiated 
a VPN connection using pluto across the public network. 

The expect script asks the uml\_netjig to signal it when all the packets 
have been played, and can then shutdown the machines properly.

The are a number of other advantages to this situation:

The startup and shutdown is under the control of the expect script. 
This means that the systems can be interrogated about the state after the
packets have occured.

As the startup/shutdown is no longer contigent on the packet flows - this
means that various tests can be done without requiring any packets to flow.

The final advantage of this system is that the expect script is only 209
lines long, of which 106 lines are argument processing. A slightly
abridged version is presented below:

\begin{verbatim}
spawn -noecho -open [open "|$netjig_prog --cmdproto" w+]

newswitch $netjig1 "$arpreply east"
newswitch $netjig1 "public"
newswitch $netjig1 "$arpreply west"

trace variable expect_out(buffer) w log_by_tracing

startuml east 
startuml west

loginuml east
loginuml west

inituml east
inituml west

record $netjig1 east $recordeast
record $netjig1 west $recordwest

setupplay $netjig1 east $playeast
setupplay $netjig1 west $playwest

runuml $netjig1

send -i $umlid(east,spawnid) "^C\r"
send -i $umlid(west,spawnid) "^C\r"

send -i $umlid(east,spawnid) "halt\r"
expect -i $umlid(east,spawnid) eof

send -i $umlid(west,spawnid) "halt\r"
expect -i $umlid(west,spawnid) eof
\end{verbatim}








\section{Conclusions}

Use of virtual testing environment massively simplies automated tests.

Limitations are that one can only test Liunx 2.4 and beyond kernels.

It takes a lot of RAM and a lot of CPU, but still is cheaper than
coordinating many physical machines.

The resulting simplication due to refactoring means that creating new 
tests is easy, and therefore many new tests will be created.




\bibliographystyle{alpha}
\bibliography{ols2002,rfc}

\end{document}
